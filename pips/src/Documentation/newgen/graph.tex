%%
%% $Id: graph.tex 23065 2016-03-02 09:05:50Z coelho $
%%
%% Copyright 1989-2016 MINES ParisTech
%%
%% This file is part of PIPS.
%%
%% PIPS is free software: you can redistribute it and/or modify it
%% under the terms of the GNU General Public License as published by
%% the Free Software Foundation, either version 3 of the License, or
%% any later version.
%%
%% PIPS is distributed in the hope that it will be useful, but WITHOUT ANY
%% WARRANTY; without even the implied warranty of MERCHANTABILITY or
%% FITNESS FOR A PARTICULAR PURPOSE.
%%
%% See the GNU General Public License for more details.
%%
%% You should have received a copy of the GNU General Public License
%% along with PIPS.  If not, see <http://www.gnu.org/licenses/>.
%%
\documentclass[a4paper]{article}

\usepackage[latin1]{inputenc}
\usepackage{newgen_domain}
\usepackage[backref,pagebackref]{hyperref}

\title{PIPS: Generic Oriented Graph}
\author{Fran�ois Irigoin \\
	Pierre Jouvelot \\
	R�mi Triolet\\
\\
	CRI, Ecole des Mines de Paris}


\begin{document}
\maketitle
\sloppy

\section*{Introduction}

The \verb+graph+ type implements an oriented graph or multigraph
(i.e. more than one arc may link two vertices). This data structure is
not really generic, i.e. types for vertices and arcs are not imported,
but types for vertex and arc {\em labels} are imported. They do not have
to be NewGen types. If they are NewGen types, they have to be handled as
non-NewGen types and explicitly dynamically declared at
initialization. When labels are used, they have to be explictly casted
into their effective type.

Since NewGen was not really designed to support re-entrance, unusual
bugs may occur when using standard or high-level NewGen primitive such
as \verb+free+ or \verb+gen_multi_recurse()+. See NewGen documentation.

The \verb+graph+ type is used for dependence graphs and for use-def
chains, with specific vertex and arc labels (see the \verb+dg+ data
structures). It should have been used for control flow graphs, but
another graph representation is embedded in the internal representation
(see \verb+unstructured+).

There is no \verb+graph-util+ package containing primitives for graph,
such as adding a vertex, removing a vertex, adding an arc, removing an
arc, printing a graph, walking a graph,.... This probably is due to the
facts that the generic graph datastructure is used only twice throughout
PIPS and that interesting algorithms like strongly connected component
computation require extra-datastructures. These datastructures are
closely linked to the concept of dependence graph because of arc levels,
and they are joined to the dependence graph specific labels.

The dependence graph, defined in \verb+dg.f.tex+, is built by the
\verb+chains+ library, updated by the \verb+ricedg+ library and used by
privatization transformations and by automatic parallelization (see the
\verb+transformation+ and \verb+rice+ libraries). Another version of the
dependence graph, the Data Flow Graph (DFG) is defined in
\verb+paf_ri.f.tex+.

\iffalse
La structure de donn�es suivante impl�mente un graphe orient�. Cette
structure est g�n�rique, mais elle peut �tre adapt�e aux besoins de
chacun gr�ce aux deux domaines externes {\tt vertex\_label} et {\tt
arc\_label} qui sont attach�s respectivement aux n{\oe}uds et aux arcs du
graphe.

Cette structure est utilis�e pour le graphe des d�pendances et aurait
d� l'�tre pour le graphe de contr�le d'un module.
\fi

\section*{Vertex and Arc Labels}

\domain{External vertex\_label}
{}

The \verb+vertex_label+ field points to the information attached to a
vertex or node. It is not defined here and must be provided by the user.

\iffalse
Le domaine {\tt vertex\_label} porte l'information attach�e aux sommets
du graphe.
\fi

\domain{External arc\_label}
{}

The \verb+arc_label+ field points to the information attached to an arc.
It is not defined here and must be provided by the user.

\iffalse
Le domaine {\tt arc\_label} porte l'information attach�e aux n{\oe}udes
du graphe.
\fi

\section*{Graph Structure}

An oriented graph is defined mathematically as a set of vertices and a
set of arcs.

\domain{Graph = vertices:vertex*}
{}

The set of vertices is implemented as a NewGen \verb+list+. It is not
clear how sets are implemented with list as long as equality is not
defined. Here, vertices only are known by their addresses and their
uniqueness is easy to check. Arcs are attached to vertices and there is
not set of arcs.

\iffalse
Un graphe se compose tout simplement de l'ensemble de tous ses n{\oe}uds,
chacun d'eux �tant repr�sent�s par le domaine {\tt vertex}. Les arcs
du graphe
sont directement attach�s aux n{\oe}uds}
\fi

\domain{Vertex = vertex\_label x successors:successor*}
{}

Each vertex is represented by an object of type \verb+vertex+. It is
identified by its address and points to its label thru the
\verb+vertex_label+ field, and to a list of arcs thru the
\verb+successors+ field. Quite unfortunately, arcs are named {\em
successor}, although they only {\em point} to successors.

The type of a \verb+vertex_label+ is assumed not to be a NewGen type. it
is application specific and must be defined somewhere else. See for
instance type \verb+dg_vertex_label+ defined in file \verb+dg.f.tex+.

\iffalse
Chaque n{\oe}ud d'un graphe est repr�sent� par un {\tt vertex}. Le
sous-domaine {\tt successors} donne la liste des successeurs du n{\oe}ud. Le
sous-domaine {\tt vertex\_label} doit �tre utilis� pour les informations
attach�es � chaque n{\oe}ud et dont la nature d�pend de l'application
utilisant le graphe; voir par exemple le domaine {\tt dg\_vertex\_label}
dans le fichier {\tt dg.f.tex}.}
\fi

\domain{Successor = arc\_label x vertex}
{}

Each arc in the graph is implemented as an object of type
\verb+successor+. The \verb+vertex+ field contains the {\em effective}
successor. The \verb+vertex_label+ field contains some information
attached to the arc. This information depends on the application. See
for instance the type \verb+dg_arc_label+ defined in file \verb+dg.f.tex+.

Note that more than one arc may link two vertices. There is no graph
primitive to add an arc. Each arc is known by its memory address. There
is no direct way to find the {\em origin} of an arc.

There is no primitive to check if a graph data structure represent a
graph or a multigraph. There is no primitive to check consistency
(e.g. each vertex pointed to by an arc in the graph vertex set).

Graphs are walked with two nested loops. The outer loop is over the
vertices. The innermost one is over each vertex edges, so-called
\verb+successors+.

\iffalse
Chaque arc d'un graphe est repr�sent� par un {\tt successor}. Le
sous-domaine {\tt vertex} contient le n{\oe}ud vers lequel l'arc pointe. Le
sous-domaine {\tt arc\_label} doit �tre utilis� pour contenir les
informations attach�es � chaque arc{,} et dont la nature d�pend de
l'application utilisant le graphe; voir par exemple le domaine {\tt
dg\_arc\_label} dans le fichier {\tt dg.f.tex}.}
\fi

\end{document}
\end
