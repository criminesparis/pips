%%
%% $Id: points_to_private.tex 21443 2012-08-16 15:55:32Z irigoin $
%%
%% Copyright 1989-2009 MINES ParisTech
%%
%% This file is part of PIPS.
%%
%% PIPS is free software: you can redistribute it and/or modify it
%% under the terms of the GNU General Public License as published by
%% the Free Software Foundation, either version 3 of the License, or
%% any later version.
%%
%% PIPS is distributed in the hope that it will be useful, but WITHOUT ANY
%% WARRANTY; without even the implied warranty of MERCHANTABILITY or
%% FITNESS FOR A PARTICULAR PURPOSE.
%%
%% See the GNU General Public License for more details.
%%
%% You should have received a copy of the GNU General Public License
%% along with PIPS.  If not, see <http://www.gnu.org/licenses/>.
%%

\documentclass{article}

\usepackage{newgen_domain}
\usepackage[backref,pagebackref]{hyperref}

\title{Points-to Analysis}
\author{Amira Mensi}
\begin{document}
\maketitle

\section{Introduction}

The data structures defined here are used to compute, store and use
points-to information. A points-to analysis is one of the pointer
analyses. It associates some abstract address to every known pointer
at any control point in the code. The description of the first
points-to analysis implemented in PIPS is in Amira Mensi's PhD
dissertation.

The points-to data structures depend on the internal representation
and on the memory effects.

\domain{import descriptor from "ri.newgen"}
{}
\domain{import approximation from "ri.newgen"}
{}
\domain{import statement from "ri.newgen"}
{}
\domain{import cell from "effects.newgen"}
{}

% These newgen data structure are used to propagate interprocedural points_to
% information. A points_to relation involved two entities :source and
% sink and an approximation of their relation.  
 

\section{Points-to arcs}
 
A points-to data structure implements an arc from an abstract memory location
called {\em source} to another abstract memory location called
{\em sink}. The arc between the two locations may either always exist,
and the approximation is EXACT, or possibly exist, and the
approximation is MAY. The abstract memory locations may contain
program variables or PIPS entities in their subscript expressions, and
their values are constrained by the {\em descriptor}.

\domain{points\_to = source:cell x sink:cell x approximation x descriptor}
{}

Note: the domains {\em approximation} and {\em descriptor} are
documented in ri.tex.

Approximations are not yet well semantically defined because they can
be related to the number of memory locations abstracted by the source,
the number of memory locations abstracted by the sink, or to the
number of arcs leaving a source. The current definition of an exact
arc is an arc starting from a unique memory location with outdegree
one and ending on a unique memory location.

Different kinds of points-to abstractions may be built. They depend on
the subsets of cells that are used, as well as on the descriptor
kind.

The initial implementation is based on {\em constant memory paths},
that are references independent of the current store and that are
called points\_to\_cells in the source code. Such references are
extensions of the references defined by \verb/ri.tex/. Field entities
can be used as subscript as well the special entity
\verb/UNBOUNDED_DIMENSION/. This is convenient for the array-based data
dependence tests and the parallelization passes of PIPS because they
can handle C \verb/struct/ variables without modifications. The API
and the source code dealing with points\_to\_cells is located in
library effects-util. No descriptor is used, and hence all descriptor
field should contain the kind \verb/none/.

Descriptors could be sets of affine constraints and be used to
represent the relation existing between arrays of pointers and
locations. For instance, \verb/a[i]->malloc[i]/ $\forall i in
[0,10[$. They could be used to represent multidimensional arrays
    allocated dynamically and to parallelize loops using them.

% \domain{access = referencing:points\_to\_path + addressing:points\_to\_path + dereferencing:points\_to\_path}{}
%\domain{points\_to\_path = reference}{}
%\domain{points\_to\_graph = arcs:points\_to*}{}

\section{Points-to relation}

The points-to relation can be represented by a list of points-to arcs:

\domain{points\_to\_list = bottom:bool x list : points\_to*}
{}

The points-to graph domain is used to represent the points-to
relation, from a set of cells to a set of cells, as a set of arcs. For
a given statement, this relation may be empty, for instance because no
pointers are used, but it may also not exist at all because the
statement is unreachable with respect to the points-to information.

\domain{points\_to\_graph =  bottom:bool x set:points\_to{}}
{}

The double representation of the points-to graph, as a list of arcs
and as a set of arcs, is linked to the implementation. The lists are
fine to store and exploit the points-to information. The set if more
efficient when the points-to information is computed. Not that the
hash function used to implement the sets of points-to arcs is
fragile. No side-effects are permitted on arcs belonging to a
points\_to\_set.

\section{Mapping from statements to their points-to relations}

This last domain is used to associate its points-to information to
each statement for storage or use of the points-to information:

\domain{statement\_points\_to = persistant statement -> points\_to\_list}
{}

The points-to information is a precondition. It holds in the memory
state preexsiting to the statement execution.

\end{document}
