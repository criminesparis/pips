%%
%% $Id: A-279-modified.tex 23065 2016-03-02 09:05:50Z coelho $
%%
%% Copyright 1989-2016 MINES ParisTech
%%
%% This file is part of PIPS.
%%
%% PIPS is free software: you can redistribute it and/or modify it
%% under the terms of the GNU General Public License as published by
%% the Free Software Foundation, either version 3 of the License, or
%% any later version.
%%
%% PIPS is distributed in the hope that it will be useful, but WITHOUT ANY
%% WARRANTY; without even the implied warranty of MERCHANTABILITY or
%% FITNESS FOR A PARTICULAR PURPOSE.
%%
%% See the GNU General Public License for more details.
%%
%% You should have received a copy of the GNU General Public License
%% along with PIPS.  If not, see <http://www.gnu.org/licenses/>.
%%

%\documentstyle[11pt,a4]{article}

%\newcommand{\mytilde}{}
%\newsavebox{\boite}
%\setbox\boite=\hbox{\verb+~+}

%\title{Array Regions for Interprocedural Parallelization and Array
%Privatization}
%\author{B�atrice {\sc Creusillet}\thanks{{\tt e-mail: <creusil@cri.ensmp.fr>,
%web: <http://www.cri.ensmp.fr/\usebox{\boite}creusil>}}\\ \\
%Centre de Recherche en Informatique \\ {\'E}cole des mines de Paris \\
%Internal Report A/279/CRI}
%\date{}

%%
%% $Id: Commandes-beatrice.tex 23065 2016-03-02 09:05:50Z coelho $
%%
%% Copyright 1989-2016 MINES ParisTech
%%
%% This file is part of PIPS.
%%
%% PIPS is free software: you can redistribute it and/or modify it
%% under the terms of the GNU General Public License as published by
%% the Free Software Foundation, either version 3 of the License, or
%% any later version.
%%
%% PIPS is distributed in the hope that it will be useful, but WITHOUT ANY
%% WARRANTY; without even the implied warranty of MERCHANTABILITY or
%% FITNESS FOR A PARTICULAR PURPOSE.
%%
%% See the GNU General Public License for more details.
%%
%% You should have received a copy of the GNU General Public License
%% along with PIPS.  If not, see <http://www.gnu.org/licenses/>.
%%

\newcommand{\references}[1]{{\em #1~references}}

\newcommand{\mybfitem}[1]{\item[{\bf#1}]}
\newcommand{\myttitem}[1]{\item[{\tt#1}]}
\newenvironment{mylist}{\begin{list}%
        {}{}}{\end{list}}

\newcommand{\smallverb}[1]{{\small \verb+#1+}}
\newcommand{\smalltt}[1]{{\small \tt #1}}

\newenvironment{csmalltt} {\begin{center} \small \tt}
                           {\end{center}}
%\newenvironment{regions} {\begin{center} \small \tt \begin{tabular}{l}}
%                           {\end{tabular} \end{center}}
\newenvironment{regions} {\[\begin{array}{l}}
                           {\end{array} \]}


\newenvironment{videitemize}{\renewcommand{\labelitemi}{}
\renewcommand{\labelitemii}{} \renewcommand{\labelitemiii}{}
\renewcommand{\labelitemiv}{}
\begin{itemize}}
{\end{itemize}}

%noms
\newcommand{\pips}[0]{{\sc pips}}
\newcommand{\PIPS}[0]{{\sc pips}}
\newcommand{\fortran}[0]{{\sc fortran}}

% pour les figures encadr{\'e}es en haut et en bas
\newcommand{\drawline}[0]{\leavevmode\hrulefill}
\newenvironment{myfigure}{\begin{figure} 
                         \drawline }
                            { \drawline 
                        \end{figure}}

\newenvironment{myfigurep}{\begin{figure}[p] 
                         \drawline }
                            { \drawline 
                        \end{figure}}

% symboles math{\'e}matiques
\newcommand{\union}[0]{\cup}
\newcommand{\U}[0]{\cup}
\newcommand{\inter}[0]{\cap}
\newcommand{\loand}[0]{\wedge}
\newcommand{\loor}[0]{\vee}
\newcommand{\tq}[0]{\:|\:}
\newcommand{\ilexiste}[0]{\exists\:}
\newcommand{\implique}[0]{\Longrightarrow}
\newcommand{\rond}[0]{\circ}
\newcommand{\meet}[0]{\wedge}
\newcommand{\Meet}[0]{\bigwedge}
\newcommand{\qqsoit}[0]{\forall \:}
\newcommand{\Umust}[0]{\cup_{must}}
\newcommand{\Umay}[0]{\cup_{may}}
\newcommand{\rminus}[0]{\circleddash}

% r{\'e}gions 
%\newcommand{\convexhull}[0]{convexhull}
%\newcommand{\must}[0]{{\tt MUST}}
%\newcommand{\may}[0]{{\tt MAY}}
\newcommand{\Tinverse}[0]{\widetilde{T}_k^{-1}}
\newcommand{\Ttilde}[2]{{\cal T}_{\sigma_{#1} \rightarrow \sigma_{#2}}}
\newcommand{\Ttildepriv}[3]{{\cal T}_{\sigma_{#1} \rightarrow
\sigma_{#2} \backslash {#3} }}
%\newcommand{\phik}[1]{\ifmmode\else$\fi\phi_{#1}\ifmmode\else$\fi}
\newcommand{\rread}[0]{{\tt  READ}}
\newcommand{\rwrite}[0]{{\tt  WRITE}}
\newcommand{\rin}[0]{{\tt   IN}}
\newcommand{\rout}[0]{{\tt OUT}}
\newcommand{\rw}[0]{{\tt READ/WRITE}}


% th{\'e}or\`{e}mes et similaires
\newtheorem{property}{Property}
\newtheorem{algorithm}{Algorithm}
%\newtheorem{definition}{Definition}
\newtheorem{notation}{Notation}
\newtheorem{criterium}{Criterium}
%\newtheorem{corollary}{Corollary}
\newenvironment{proff}{\paragraph{Proof:}}
                        { {\hfill\ \hfill \(\square\)} \paragraph{} }



\newcommand{\matheq}[1]{$#1$}


\newcommand{\tab}{\hspace{5mm}}


%\begin{document}

%\maketitle 

%\begin{abstract}
%  \em Three demonstrations are presented, that highlight the need for
%  interprocedural analyses such as {\em preconditions\/} and exact {\em
%  array regions\/}, in order to parallelize loops that contain subroutine
%  calls or temporary arrays. These analyses are provided by PIPS in an
%  interactive environment.
%\end{abstract}


\subsection{Interprocedural Parallelization}

\verb+AILE+ is an application from the ONERA, the French institute of
aerospatial research. It has more than 3000 lines of FORTRAN code. It has
been slightly modified to test the coherence of some input values.

The aim of this demonstration is to show that interprocedural analyses are
necessary for an automatic parallelization. 


For that purpose, we have chosen the subroutine (or {\em module\/})
\verb+EXTR+, which is called by the module \verb+GEOM+, itself called by
the main routine \verb+AILE+. An excerpt is given in Figure~\ref{fig:AILE}
(without the intermediate call to \verb+GEOM+).


\begin{figure}[htbp] 
  \begin{center} \footnotesize
    \leavevmode
\begin{minipage}{12cm}
\begin{verbatim}
      PROGRAM AILE
      DIMENSION T(52,21,60)
      COMMON/CT/T
      COMMON/CI/I1,I2,IMAX,I1P1,I1P2,I2M1,I2M2,IBF
      COMMON/CJ/J1,J2,JMAX,J1P1,J1P2,J2M1,J2M2,JA,JB,JAM1,JBP1
      COMMON/CK/K1,K2,KMAX,K1P1,K1P2,K2M1,K2M2
      COMMON/CNI/L      
      ...
      READ(NXYZ) I1,I2,J1,JA,K1,K2
C     
      IF(J1.GE.1.AND.K1.GE.1) THEN
         N4=4
         J1=J1+1
         J2=2*JA+1
         JA=JA+1
         K1=K1+1
         ...
         CALL EXTR(NI,NC)
      ENDIF
      END

      SUBROUTINE EXTR(NI,NC)
      DIMENSION T(52,21,60)
      COMMON/CT/T
      COMMON/CI/I1,I2,IMAX,I1P1,I1P2,I2M1,I2M2,IBF
      COMMON/CJ/J1,J2,JMAX,J1P1,J1P2,J2M1,J2M2,JA,JB,JAM1,JBP1
      COMMON/CK/K1,K2,KMAX,K1P1,K1P2,K2M1,K2M2
      COMMON/CNI/L
      L=NI
      K=K1
      DO 300 J=J1,JA
         S1=D(J,K  ,J,K+1)
         S2=D(J,K+1,J,K+2)+S1
         S3=D(J,K+2,J,K+3)+S2
         T(J,1,NC+3)=S2*S3/((S1-S2)*(S1-S3))
         T(J,1,NC+4)=S3*S1/((S2-S3)*(S2-S1))
         T(J,1,NC+5)=S1*S2/((S3-S1)*(S3-S2))
         JH=J1+J2-J
         T(JH,1,NC+3)=T(J,1,NC+3)
         T(JH,1,NC+4)=T(J,1,NC+4)
         T(JH,1,NC+5)=T(J,1,NC+5)
 300  CONTINUE      
      END

      REAL FUNCTION D(J,K,JP,KP)
      DIMENSION T(52,21,60)
      COMMON/CT/T
      COMMON/CNI/L
C     
      D=SQRT((T(J,K,L  )-T(JP,KP,L  ))**2
     1     +(T(J,K,L+1)-T(JP,KP,L+1))**2
     2     +(T(J,K,L+2)-T(JP,KP,L+2))**2)
      END

\end{verbatim}
  \end{minipage}
  \end{center}
  \caption{Excerpt from program {\tt AILE}.}
  \label{fig:AILE}
\end{figure}

\subsubsection{EXTR}

\verb+EXTR+ contains a \verb+DO+ loop that has several characteristics:
\begin{enumerate}
\item There are several read and write references to elements of the array
  \verb+T+. This induces dependences that cannot be disproved if we don't
  know the relations between index expressions, and more precisely between
  \verb+J+ and \verb+JH+. We already know that \verb|JH=J1+J2-J|, but we
  don't know the values of \verb+J1+, \verb+J2+ and \verb+JA+, which are
  global variables initialized in \verb+AILE+. Thus, we can disprove the
  loop-carried dependences between \verb|T(J,1,NC+3)| and
  \verb|T(JH,1,NC+3)| for instance, only if we interprocedurally propagate
  the values of \verb+J1+, \verb+J2+ and \verb+JA+ from \verb+AILE+. This
  type of information is called {\em precondition\/} in PIPS~\cite{Irig:91,cnrs-nsf92:pips}.

\item There are three calls to the function \verb+D+ in \verb+EXTR+.
  \verb+D+ contains several read references to the global array \verb+T+.
  So, we must assume that the whole array is potentially read by each call
  to \verb+D+. This induces dependences in \verb+EXTR+ between the calls to
  \verb+D+ and the other statements. In order to disprove these dependences,
  we need a way to represent the set of array elements read by any
  invocation of \verb+D+, and be able to use this information at each call
  site. These sets are called {\em array regions\/} in PIPS~\cite{Creu:95a}.

\item \verb+S1+, \verb+S2+, \verb+S3+ and \verb+JH+ are defined and used at
  each iteration. This induces loop-carried dependences. But we may notice
  that each use is preceded by a definition in the same iteration. These
  variables can be privatized (this means that a local copy is assigned to
  each iteration) to remove the spurious dependences.
\end{enumerate}


\subsubsection{D}

As written before, there are several references to elements of the array
\verb+T+ in \verb+D+. Our aim is to represent this set of elements, such
that it can be used at each call site to help disproving dependences. 

If we know nothing about the relations between the values of \verb+K+ and
\verb+KP+ or between \verb+J+ and \verb+JP+, all we can deduce is that the
third index of all the array elements ranges between \verb+L+ and
\verb|L+2|. This is represented by the region:
\begin{regions} 
<T(\phik{1},\phik{2},\phik{3})-R-MAY-\{L<=\phik{3}<=L+2\}>
\end{regions}
The $\phi$ variables represent the dimensions of the array; \verb+R+ means
that we consider the {\em read\/} effects on the variable; and \verb+MAY+
means that the region is an over-approximation of the set of elements that
are actually read.

The relations between the values of \verb+K+ and \verb+KP+ or \verb+J+ and
\verb+JP+ are those that exist between the real arguments. At each call
site, we have {\tt JP==J} and {\tt KP==K+1}. These contidions hold true
before each execution of {\tt D}; we call them {\em preconditions\/}. Under
these conditions, we can now recompute the region associated to the array
{\tt T}:
\begin{regions} 
<T(\phik{1},\phik{2},\phik{3})-R-MUST-\{\phik{1}==J, K<=\phik{2}<=K+1, L<=\phik{3}<=L+2\}>
\end{regions}
Notice that this is a {\tt MUST} region, because it exactly represents the
set of array elements read by any invocation of function {\tt D}. 

\subsubsection{Parallelisation of EXTR}

We can now parallelize {\tt EXTR} by:
\begin{enumerate}
\item privatizing the scalar variables;
\item using {\em array regions\/} to summarize the read effects on the array
  {\tt T} by each invocation of {\tt D};
\item using the {\em preconditions\/} induced by the initialization of
  global scalar variables (in {\tt AILE}) to disprove the remaining dependences.
\end{enumerate}
This leads to the parallelized version of Figure~\ref{fig:EXTR_para}.

\begin{figure}[htbp]
  \begin{center} \footnotesize
    \leavevmode
    \begin{minipage}{12cm}
\begin{verbatim}
      SUBROUTINE EXTR(NI,NC)
      DIMENSION T(52,21,60)
      COMMON/CT/T
      COMMON/CI/I1,I2,IMAX,I1P1,I1P2,I2M1,I2M2,IBF
      COMMON/CJ/J1,J2,JMAX,J1P1,J1P2,J2M1,J2M2,JA,JB,JAM1,JBP1
      COMMON/CK/K1,K2,KMAX,K1P1,K1P2,K2M1,K2M2
      COMMON/CNI/L
      L = NI                                                        
      K = K1                                                        
      DOALL J = J1, JA
         PRIVATE S1,S2,S3
         S1 = D(J, K, J, K+1)                                       
         S2 = D(J, K+1, J, K+2)+S1                                  
         S3 = D(J, K+2, J, K+3)+S2                                  
         T(J,1,NC+3) = S2*S3/((S1-S2)*(S1-S3))                      
         T(J,1,NC+4) = S3*S1/((S2-S3)*(S2-S1))                      
         T(J,1,NC+5) = S1*S2/((S3-S1)*(S3-S2))                      
      ENDDO
      DOALL J = J1, JA
         PRIVATE JH
         JH = J1+J2-J                                               
         T(JH,1,NC+3) = T(J,1,NC+3)                                 
         T(JH,1,NC+4) = T(J,1,NC+4)                                 
         T(JH,1,NC+5) = T(J,1,NC+5)                                 
      ENDDO
      END
\end{verbatim}
    \end{minipage}
  \end{center}
  \caption{Parallelized version of {\tt EXTR}.}
  \label{fig:EXTR_para}
\end{figure}


\subsection{Array Privatization}

Array privatization is not yet implemented in PIPS, but the information
needed to perform the transformation is already available: {\em IN and OUT
regions}~\cite{Creu:95a,Creu:95b}.

To illustrate the characteritics of these regions, we will consider two
examples: {\tt NORM} is another excerpt from {\tt AILE}, and {\tt RENPAR6}
is a contrived example that highlights some details of the computation of
regions and the possibilities opened up by IN and OUT regions.


\subsubsection{NORM}


\begin{figure}[htbp]
  \begin{center} \footnotesize
    \leavevmode
    \begin{minipage}{12cm}
\begin{verbatim}
      PROGRAM AILE
      DIMENSION T(52,21,60)
      COMMON/CT/T
      COMMON/CI/I1,I2,IMAX,I1P1,I1P2,I2M1,I2M2,IBF
      COMMON/CJ/J1,J2,JMAX,J1P1,J1P2,J2M1,J2M2,JA,JB,JAM1,JBP1
      COMMON/CK/K1,K2,KMAX,K1P1,K1P2,K2M1,K2M2
      COMMON/CNI/L
      DATA N1,N3,N4,N7,N10,N14,N17/1,3,4,7,10,14,17/

      READ(NXYZ) I1,I2,J1,JA,K1,K2
C     
      IF(J1.GE.1.AND.K1.GE.1) THEN
         N4=4
         J1=J1+1
         J2=2*JA+1
         JA=JA+1
         K1=K1+1
         CALL NORM(N10,N7,N4,N14,N17,I2)
      ENDIF
      END

      SUBROUTINE NORM(LI,NI,MI,NN,NC ,I) 
      DIMENSION T(52,21,60)
      DIMENSION TI(3)

      COMMON/T/T
      COMMON/I/I1,I2,IMAX,I1P1,I1P2,I2M1,I2M2,IBF
      COMMON/J/J1,J2,JMAX,J1P1,J1P2,J2M1,J2M2,JA,JB,JAM1,JBP1
      COMMON/K/K1,K2,KMAX,K1P1,K1P2,K2M1,K2M2
      COMMON/IO/LEC ,IMP,KIMP,NXYZ,NGEO,NDIST

C ....
DO 300 K=K1,K2
      DO 300 J=J1,JA

      CALL PVNMUT(TI)
      T(J,K,NN  )=S*TI(1)
      T(J,K,NN+1)=S*TI(2)
      T(J,K,NN+2)=S*TI(3)
  300 CONTINUE
C ....
      END

      SUBROUTINE PVNMUT(C)
      DIMENSION C(3), CX(3)
      CX(1)= 1
      CX(2)= 2
      CX(3)= 3
      R=SQRT(CX(1)*CX(1)+CX(2)*CX(2)+CX(3)*CX(3))
      IF(R.LT.1.E-12) R=1.
      DO I = 1,3
      C(I) = CX(I)/R
      ENDDO
      RETURN
      END
\end{verbatim}
    \end{minipage}
  \end{center}
  \caption{Another excerpt from {\tt AILE}: {\tt NORM}}
  \label{fig:NORM}
\end{figure}

This is a very simple example (see Figure~\ref{fig:NORM}) that shows the
necessity of array privatization, and the need for IN and OUT array regions.

In the loop of subroutine {\tt NORM}, the references to the array {\tt T} do
not induce loop-carried dependences.  Furthermore, there are only read-read
dependences on {\tt S}.  However, notice that the array {\tt TI} is a real
argument in the call to {\tt PVNMUT}, and that there are 3 read references
to array {\tt TI}.  This induces potential interprocedural dependences. We
have seen with the previous example that these dependences can sometimes be
disproved with array regions.

We must first compute the regions of array {\tt TI} that are referenced in
{\tt {\tt PVNMUT}}. In {\tt PVNMUT}, {\tt TI} is called {\tt C}. And the 3
elements of {\tt C} are written, but not read. This leads to:
\begin{regions}
<C(\phik{1})-W-MUST-\{1<=\phik{1}<=3\}>
\end{regions}
({\tt W} means that this  is a {\em write\/} effect)

At the call site, {\tt C} is translated into {\tt TI}, which gives the
region:
\begin{regions}
<TI(\phik{1})-W-MUST-\{1<=\phik{1}<=3\}>
\end{regions}
And finally, the regions corresponding to the whole body of the loop nest
are:
\begin{regions}
<TI(\phik{1})-W-MUST-\{1<=\phik{1}<=3\}> \\
<TI(\phik{1})-R-MUST-\{1<=\phik{1}<=3\}>
\end{regions}

These regions are identical, which means that each iteration of loops {\tt K} and
{\tt J} reads and writes to the same memory locations of array {\tt TI}. Thus, there are
loop-carried dependences, and the loop cannot be parallelized.

However, these dependences are false dependences, because if we allocate a
copy of array {\tt TI} to each iteration (in fact to each processor), there are no
more dependences. This is what is called array privatization. In order to
privatize an array, we must be sure that, in each iteration, no element is
read before being written in the same iteration. Thus, there are no
loop-carried producer-consumer dependences.  

This last property cannot be verified by using READ regions, because they
contain all the elements that are read, and not only those that are read
before being written. This is represented in PIPS by IN regions. In our case,
we must verify that no element of {\tt TI} belongs to the IN region corresponding
to the loop body, which is the case. 

We must also be sure that no element of {\tt TI} that is initialized by a single
iteration is used in the subsequent iterations or after the loops.
This information is provided in PIPS by the OUT regions. They represent
the set of live array elements, that is to say those that are used in
the continuation.

We can now parallelize {\tt NORM} by:
\begin{enumerate}
\item using {\em array regions\/} to perform the dependence analysis;
\item using {\em IN\/} and {\em OUT array regions\/} to privatize the array
  {\tt TI}.
\end{enumerate}


This leads to the parallelized version of Figure~\ref{fig:NORM_para}.

\begin{figure}[htbp]
  \begin{center} \footnotesize
    \leavevmode
    \begin{minipage}{12cm}
\begin{verbatim}
      SUBROUTINE NORM(LI,NI,MI,NN,NC ,I) 
      DIMENSION T(52,21,60)
      DIMENSION TI(3)

      COMMON/CT/T
      COMMON/I/I1,I2,IMAX,I1P1,I1P2,I2M1,I2M2,IBF
      COMMON/J/J1,J2,JMAX,J1P1,J1P2,J2M1,J2M2,JA,JB,JAM1,JBP1
      COMMON/K/K1,K2,KMAX,K1P1,K1P2,K2M1,K2M2
      COMMON/IO/LEC ,IMP,KIMP,NXYZ,NGEO,NDIST

C     ....
      DOALL K = K1, K2
         PRIVATE J
         DOALL J = J1, JA
            PRIVATE TI
            CALL PVNMUT(TI)                                             
            T(J,K,NN) = S*TI(1)                                         
            T(J,K,NN+1) = S*TI(2)                                       
            T(J,K,NN+2) = S*TI(3)                                       
         ENDDO
      ENDDO
C     ....
      END
\end{verbatim}
    \end{minipage}
  \end{center}
  \caption{Parallelized version of {\tt NORM}.}
  \label{fig:NORM_para}
\end{figure}

\subsubsection{RENPAR6}


\begin{figure}[htbp]
  \begin{center} \footnotesize
    \leavevmode
    \begin{minipage}{12cm}
\begin{verbatim}
      SUBROUTINE RENPAR6(A,N,K,M)
      INTEGER N,K,M,A(N)
      DIMENSION WORK(100,100)
      K = M * M
      DO I = 1,N
         DO J = 1,N
            WORK(J,K) = J + K
         ENDDO

         CALL INC1(K)

         DO J = 1,N
            WORK(J,K) = J * J - K * K
            A(I) = A(I) + WORK(J,K) + WORK(J,K-1)
         ENDDO
      ENDDO
      END
      
      SUBROUTINE INC1(I)
      I = I + 1
      END
\end{verbatim}
    \end{minipage}
  \end{center}
  \caption{Contrived example: {\tt RENPAR6}}
  \label{fig:RENPAR6}
\end{figure}

{\tt RENPAR6} is a contrived example (see Figure~\ref{fig:RENPAR6}) designed
to show on a very simple program the power of READ, WRITE, IN and OUT
regions, and some particular details of their computations, especially when
integer scalar variables that appear in array indices are modified.

The main purpose is to see that array {\tt WORK} is only a temporary and can
be privatized.  Notice that the value of {\tt K} is unknown on entry to the
loop {\tt I}, and that its value is modified by a call to {\tt INC1} at each
iteration ({\tt INC1} simply increments its value by 1).


We are interested in the sets of array elements that are referenced in each
iteration. However, since the value of {\tt K} is not the same in the two written
references, we cannot summarize the write accesses if we do not know the
relation that exists between the two values of {\tt K}. This is achieved in PIPS
by using transformers, that here show how the new value of {\tt K} is related to the
value before the {\tt CALL} ({\tt K\#init}):
\begin{regions}
T(K) \{K==K\#init+1\}
\end{regions}
And the transformer of the loop shows how the value of {\tt K} at each step
is related to the values of {\tt I} and {\tt K\#init} (value of {\tt K}
before the loop):
\begin{regions}
T(I,K) \{K==I+K\#init-1\}
\end{regions}

This previous information is used to summarize the sets of elements that are
read or written by each program structure. In order to compute the summary
for the loop {\tt I}, we must merge the sets for the two {\tt J} loops. Be
careful that the value of {\tt K} is not the same for these two loops. We
must use the transformer of the {\tt CALL} to translate the value of {\tt K}
in the second region into the value of {\tt K} before the CALL. At this
step, we have a summary of what is done by a single iteration. We then
compute the regions for the whole loop {\tt I}. This is done with the help
of the transformer of the loop that gives the relation between {\tt K} and
{\tt I}.

However, as we have seen with {\tt NORM}, READ and WRITE regions are not
sufficient for array privatization, because we must verify that every
element of {\tt WORK} that is read by an iteration is previously written in
the same iteration. This is achieved by the IN region.  Then OUT regions
allow us to verify that no element of {\tt WORK} is used in the subsequent
iterations or in the continuation of the loop. 




We can now try to parallelize {\tt RENPAR6} by:
\begin{enumerate}
\item using {\em transfomers\/} to compute {\em array regions\/};
\item using {\em array regions\/} to perform the dependence analysis;
\item using {\em IN\/} and {\em OUT array regions\/} to privatize the array
  {\tt WORK}.
\end{enumerate}


This leads to the parallelized version of Figure~\ref{fig:RENPAR6_para}. The
array {\tt WORK} is privatized in loop {\tt I}. However, the loop is not
parallelized, because automatic induction variable substitution is not
available in PIPS. This transformation has been performed by hand. This
leads to the subroutine {\tt RENPAR6\_2} in figure~\ref{fig:RENPAR6_2}.
And after array privatization, PIPS is able to parallelize the loop {\tt I}
(see Figure~\ref{fig:RENPAR6_2_para}).

\begin{figure}[htbp]
  \begin{center} \footnotesize
    \leavevmode
    \begin{minipage}{12cm}
\begin{verbatim}
      SUBROUTINE RENPAR6(A,N,K,M)
      INTEGER N,K,M,A(N)
      DIMENSION WORK(100,100)
      K = M*M                                                      
      DO I = 1, N
         PRIVATE WORK,I
         DOALL J = 1, N
            PRIVATE J
            WORK(J,K) = J+K                                         
         ENDDO
         CALL INC1(K)                                              
         DOALL J = 1, N
            PRIVATE J
            WORK(J,K) = J*J-K*K                                     
         ENDDO
         DO J = 1, N
            PRIVATE J
            A(I) = A(I)+WORK(J,K)+WORK(J,K-1)                        
         ENDDO
      ENDDO
      END
\end{verbatim}
    \end{minipage}
  \end{center}
  \caption{Parallelized version of {\tt RENPAR6}.}
  \label{fig:RENPAR6_para}
\end{figure}


\begin{figure}[htbp]
  \begin{center} \footnotesize
    \leavevmode
    \begin{minipage}{12cm}
\begin{verbatim}
      SUBROUTINE RENPAR6_2(A,N,K,M)
      INTEGER N,K,M,A(N)
      DIMENSION WORK(100,100)
      K0 = M * M
      DO I = 1,N
         K = K0+I-1
         DO J = 1,N
            WORK(J,K) = J + K
         ENDDO

         CALL INC1(K)

         DO J = 1,N
            WORK(J,K) = J * J - K * K
            A(I) = A(I) + WORK(J,K) + WORK(J,K-1)
         ENDDO
      ENDDO
      END
\end{verbatim}
    \end{minipage}
  \end{center}
  \caption{{\tt RENPAR6\_2}.}
  \label{fig:RENPAR6_2}
\end{figure}

\begin{figure}[htbp]
  \begin{center} \footnotesize
    \leavevmode
    \begin{minipage}{12cm}
\begin{verbatim}
      SUBROUTINE RENPAR6_2(A,N,K,M)
      INTEGER N,K,M,A(N)
      DIMENSION WORK(100,100)
      K0 = M*M                                                 
      DOALL I = 1, N
         PRIVATE WORK,J,K,I
         K = K0+I-1                                                
         DOALL J = 1, N
            PRIVATE J
            WORK(J,K) = J+K                                        
         ENDDO
         CALL INC1(K)                                               
         DOALL J = 1, N
            PRIVATE J
            WORK(J,K) = J*J-K*K                                       
         ENDDO
         DO J = 1, N
            PRIVATE J
            A(I) = A(I)+WORK(J,K)+WORK(J,K-1)                         
         ENDDO
      ENDDO
      END
\end{verbatim}
    \end{minipage}
  \end{center}
  \caption{Parallelized version of {\tt RENPAR6\_2}.}
  \label{fig:RENPAR6_2_para}
\end{figure}






In fact, IN and OUT regions could also be used to reduce the set of elements
of array {\tt WORK} to allocate to each processor, because each iteration
only accesses  a sub-array. These regions provide an exact representation of the
set of elements that are actually needed.

%{\small
%\bibliographystyle{plain}
%\bibliography{mybib}

%}\end{document}






